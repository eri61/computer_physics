\documentclass[dvipdfmx, titlepage]{jsarticle}   %日本語対応
\bibliographystyle{junsrt}
%
\usepackage[utf8]{inputenc}     %欧文でutf-8対応にする
\usepackage[hang,small,bf]{caption}
\usepackage[subrefformat=parens]{subcaption}
\usepackage{amsmath,amssymb}    %数式
\usepackage{bm}                 %太字ベクトル
\usepackage{graphicx}           %図の挿入
\usepackage{here}               %図の[H]を使えるようにする
\usepackage{ascmac}             %枠を使う
\usepackage{physics}            %コマンドの略称を用いる
\usepackage{comment}            %コメントの使用
\usepackage{enumerate}          %箇条書きを可能に
\usepackage{url}                %urlの利用
\usepackage{listings,jvlisting}


\begin{document}

\begin{titlepage}

  \begin{center}
    \vspace*{180truept}
    \huge コンピュータ物理学
    \footnotesize \vskip\baselineskip
    \Large   3次元イジング模型 \\%subtitle
    \vspace{70truept}
    \Large
    \renewcommand{\arraystretch}{1.2}
    \begin{tabular}{rl}
      学生番号 & 05502231 \\
      氏名     & 松田愛理
    \end{tabular}
    \vspace{8pt}\linebreak
    \date{2023年1月17日} \linebreak
    \vspace{50pt}
    \vskip\baselineskip

  \end{center}
\end{titlepage}


\section{背景・目的}
イジング模型とは、原子磁石の相互的な振る舞いから巨視的な磁石の振る舞いを記述する最も簡単なモデルの一つである。今回はこのイジング模型を、1次元、2次元、3次元の3通りについて解き、次元を上げることによって変化する物理量を調べることを目的とする。また、計算を行うプログラミング言語にはpythonを用い、pythonの特性から数値解を得る方法としてテンソルネットワークを導入する。なお、1次元、2次元のイジング模型を解く際には、解の導出の他、「解析的な方法から得られた厳密解」と「テンソルネットワークにより得られた数値解」を比較し、数値解の厳密性の評価を行うこととする。

\subsection{問題設定}

\section{計算手法}

\section*{実装}

\section{計算結果}
\subsection{1次元イジングモデル}
\subsection{2次元イジングモデル}
\subsection{3次元イジングモデル}


\section{考察}

\section{結論}

\begin{thebibliography}{99}
  \bibitem{comp-ryoushi} パーソナルコンピューターを用いた量子力学入門、 桜井 捷海、裳華房(1989)
  \bibitem{tensor}SGCライブラリ169 『 テンソルネットワークの起訴と応用』、西野友年、サイエンス社(2021)
\end{thebibliography}


\end{document}